\secnewpage{Convergerende reeksen}
\subsection{Reeksen}
\paragraph{Wat is een reeks?} Gegeven een rij \(\downsequence{a}\), is de partiële som \(\downsequence{s}\) de volgende som:
\[
    s_{n}\defeq\bsum{i=1}{n}a_{i}
\]
Dan definiëren we de oneindige som als volgt:
\[
    \bsum{i=1}{\infty}a_{n}\defeq\liminfty{n}(s_{n}).
\]
We zeggen dat deze som convergeert als deze limiet convergeert.

\subparagraph{Cauchy criterium} Zij \(\sum a_{n}\) een reeks. Dan voldoet deze aan het Cauchy criterium als geldt dat de rij van de partiële sommen \(s_{n}\) voldoet aan het Cauchy criterium:

\begin{quotation}
    Voor alle \(\epsilon>0\) is er een \(N\) zodat voor alle \(n,m>N\) geldt dat \(\left|s_{n}-s_{m}\right|<\epsilon\).
\end{quotation}

Omdat deze definitie symmetrisch is over \(n\) en \(m\) kunnen we aannemen dat \(n\geq m\), en dus geldt dat \(|s_{n}-s_{m}|=\left|\bsum{i=m}{i=n}a_{n}\right|<\epsilon\).

Een rij convergeert dan en slechts dan als het voldoet aan het Cauchy criterium.

\subparagraph{Convergentie van reeksen en rijen} Zij \(\sum a_{n}\) een reeks die convergeert, dan geldt dat\\\(\liminfty{n}a_{n}=0\).

\subsection{Convergentie testen}
Er zijn verschillende manieren om convergentie van een reeks te bepalen, hier volgen nu enkele simpele manieren:
\paragraph{Vergelijkingstest} Gegeven een rij \(\downsequence{a}\) van positieve getallen gelden de volgende twee dingen:
\begin{items}
    \item Als voor een rij \(\downsequence{b}\) geldt dat \(|b_{n}|\leq a_{n}\) voor alle \(n\in\n\) dan geldt dat \(\sum b_{n}\) convergeert.
    \item Als voor een rij \(\downsequence{b}\) geldt dat \(b_{n}\geq a_{n}\) en \(\sum a_{n}=\infty\), dan geldt dat \(\sum b_{n}=\infty\).
\end{items}

\paragraph{Absolute convergentie} Zij \(\downsequence{b}\) een rij zodat geldt dat de reeks \(\sum|b_{n}|\) convergeert, dan geldt dat \(\sum b_{n}\) ook convergeert.

\paragraph{Verhoudingscriterium} Zij \(\downsequence{a}\) een rij. Dan geldt dat:
\begin{enum}[i.]
    \item De reeks \(\sum a_{n}\) convergeert als \(\liminfty{n}\left|\frac{a_{n+1}}{a_{n}}\right|\).
    \item De reeks \(\sum a_{n}\) divergeert als \(\liminfty{n}\left|\frac{a_{n+1}}{a_{n}}\right|>1\).
    \item Als geldt dat \(\liminfty{n}\left|\frac{a_{n+1}}{a_{n}}\right|=1\), dan is er niks te zeggen over de convergentie van de reeks.
\end{enum}

\paragraph{Wortercriterium} Zij \(\downsequence{a}\) een rij. Dan geldt dat:
\begin{enum}[i.]
    \item Als \(\limsup\left|a_{n}\right|^{\frac{1}{n}}<1\), dan convergeert de reeks \(\sum a_{n}\).
    \item Als \(\limsup\left|a_{n}\right|^{\frac{1}{n}}>1\), dan divergeert de reeks \(\sum a_{n}\).
    \item Net als bij het verhoudingscriterium geldt dat als \(\limsup\left|a_{n}\right|^{\frac{1}{n}}=1\), dat de test dan geen uitslag geeft.
\end{enum}

\paragraph{Integraaltest} Zij \(f\colon\r\to\r\) een integreerbare functie, dan is \(\sum_{n=N}^{\infty}f(n)\) een oneindige reeks. Als geldt dat \(\int_{N}^{\infty}f(x)\,dx=\infty\), dan geldt dat de reeks \(\sum_{n=N}^{\infty}f(b)\) ook divergeert naar \(\infty\). Dit is omdat \(\int_{N}^{\infty}f(x)\,dx=\infty\leq\sum_{n=N}^{\infty}f(n)\)

Als geldt dat \(\int_{N}^{\infty}f(x)\,dx\) convergeert, dan geldt dat de reeks \(\sum_{n=N}^{\infty}f(x)\) ook convergeert. Dit is omdat \(\sum_{N}^{\infty}f(x)<f(N)+\int_{N}^{\infty}f(x)\,dx\).

\paragraph{Alternerende reeksen} Zij \(\downsequence{a}\) een dalende rij met \(\liminfty{n}a_{n}=0\). Dan geldt dat de reeks \(\sum(-1^{n+1}a_{n})\) convergeert.