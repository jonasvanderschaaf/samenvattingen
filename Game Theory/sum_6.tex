\begin{definition}[Stategy]
    A strategy in an extensive game tells you what a player does for every information set where she has the move. It it gives one action for every information set.
\end{definition}


\begin{definition}[Weak sequential equilibrium]
    We want to want to determine which equilibrium is better. We force each player whose turn it is to move forms a belief about the histories in her
    information set. A belief system is a collection of beliefs, one for each information set
    We are working with behavioral strategy in extensive game. This is a probability function that assigns to each information set a
    probability distribution over the possible actions in that set. Behavioral strategies are equivalent to mixed strategies, but easier to work with
\end{definition}


\begin{example}[card game]
    \textbf{Behavioral strategy: } With behavioral strategy, you give a probability distribution for $\{Raise, See\}$ after High card
    other probability distribution for $\{Raise, See\}$ after Low card\\
    \textbf{Mixed strategy: } Using mixed strategy, we make one probability distribution for the set
    $\{(Raise,Raise), (Raise, See), (See, Raise), (See,See)\}$
\end{example}


\begin{definition}[Weak sequential equilibrium](Kreps and Wilson, 1982)
    A Weak sequential equilibrium, also know as a Perfect Bayesian Equilibrium is an assessment, that is a behavioral strategy profile $\beta$ and a belief system $\mu$, that satisfies
    \begin{enumerate}[label=(\roman*)]
        \item \textbf{Sequential rationality: } Each player’s strategy is optimal in the part of the game that follows each of her
              information sets, given the strategy profile and her belief about the history in the
              information set that has occurred.
        \item \textbf{Weak consistency of beliefs with strategies: }
              For every Information Set reached with positive probability given the strategy profile $\beta$,
              the probability assigned by the belief system to each history $h^*$ in the information set $I_i$ is
              given by Bayes’ rule. That is:
              \begin{equation*}
                  \pp[ h^* \mid I_i ] = \frac{\pp[ h^* \text{ according to } \beta ]}{\sum_{h \in I_i}\pp [h \text{ according to } \beta ]}
              \end{equation*}
              If an information set is not reached with positive probability, any belief is allowed.
    \end{enumerate}
\end{definition}


\begin{definition}[Signaling games] (Spence, 1973)
    \begin{itemize}
        \item Why do gazelles engage in stotting when approached by cheetah?
        \item Why do men give expensive flowers to women?
        \item Why do students follow a costly but (sometimes) useless education?
    \end{itemize}
    To signal their quality.

    \begin{itemize}
        \item In signaling game, first mover has private information that is relevant to second mover
        \item First mover wants to convince second mover to choose particular beneficial action
        \item “Good” types may send costly signals to distinguish themselves from “bad” types
              \begin{itemize}
                  \item Signaling games have \textbf{separating equilibria}, in which different types choose
                        different signals and second movers infer types from signals
                  \item Signaling games have \textbf{pooling equilibria}, in which different types choose the same
                        signal and second movers do not update their beliefs
              \end{itemize}
    \end{itemize}
\end{definition}