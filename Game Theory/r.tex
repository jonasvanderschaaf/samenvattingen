\subsection{Some remarks on the exam}
\begin{itemize}
\item The final exam will be a three-hour written examination with three open questions. One question will be relatively simple, one will be
relatively difficult and the other is in between. Simple questions have a rating of 1 and
difficult questions a rating of 3. 
\item The sum of the ratings of the questions at the exam will approximately be 6.0. 
\item The probability that you get (exactly) one of the questions of Osborne is 0. Still, it will help
you a lot if you practice those questions. Try to actively make the questions yourself.
My judgment is that if you feel comfortable with the questions dealt with at the tutorials and
the material presented on the slides, you will do well on the exam. It is not necessary to
practice other questions, but you can of course always do this. Then you are best advised to
practice questions with a black circle, so that you can compare your answers with the ones
Osborne provides on his website.
\item If your answers are in the ``same spirit'' as the ones that you received during the tutorials or
the ones that you can find on the website of Osborne, you are doing well. However, in game
theory questions often have multiple solutions, so usually there is more than one way to earn
the maximum number of points. Some people prefer to avoid notation as much as possible
and to use words instead. If you are one of those people, try to be as precise as possible in
words.
\end{itemize}

\subsection{Additional Material}
\begin{itemize}
\item \href{https://www.youtube.com/watch?v=2d_dtTZQyUM}{Nash Equilibrium (from A Beautiful Mind)}
\end{itemize}