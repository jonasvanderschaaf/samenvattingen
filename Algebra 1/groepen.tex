\secnewpage{Groepen die misschien handig zijn}
\subsection{De groepen  \texorpdfstring{\(\z/n\z\)}{z/nz} en \texorpdfstring{\((\z/n\z)^{*}\)}{z/nz*}}
De groep \(\z^{+}=\z\) is abels, dus elke ondergroep is een normaaldeler. Elke ondergroep van \(\z\) is van de vorm \(n\z\). Dan kun je de factorgroep \(\z/n\z\) construeren. De groep \(\z/n\z\) heeft orde \(n\).

Je vermenigvuldiging uitvoeren in de restklassen, dan geldt dat
\[
    (\z/n\z)^{*}=\varphi(n)\defeq\#\{m\mid\ggd(n,m)=1\}.
\]

Voor \(\z/n\z\) geldt dat \(\Aut(\z/n\z)\cong(\z/n\z)^{*}\).

Alle conjugatieklassen hebben precies \(1\) element, en \(Z(G)=G\) want \(G\) is abels.

\subsection{Quaternionen}
\paragraph{Optelling} De verzameling van quaternionen \(\h^{+}\) is als volgt gedefinieerd:
\[
    \h^{+}\defeq\{a+bi+cj+dk\mid a,b,c,d\in\r\}.
\]
met de bewerking
\[
    (a+bi+cj+dk)+(a'+b'i+c'j+d'k)=(a+a')+(b+b')i+(c+c')j+(d+d')k.
\]
Dit vormt een groep.

\paragraph{Vermenigvuldiging} De quaternionen kunnen ook vermenigvuldigd worden. De groep \(\h^{*}\) is de groep quaternionen met vermenigvuldiging en is als volgt gedefinieerd:
\[
    \h^{*}\defeq\h^{+}\setminus\{0\},
\]
de vermenigvuldiging gaat volgens de volgende regels:
\begin{align*}
    i^{2}=j^{2}=k^{2}=-1, \\
    ij=k,\>ji=-k,         \\
    jk=i,\>kj=-i,         \\
    ki=j,\>ik=-j.
\end{align*}
Deze vermenigvuldiging vormt ook een groep. Deze vermenigvuldiging is niet commutatief.

\subsection{Viergroep van Klein}
De viergroep van Klein is een groep van orde \(4\) met \(V_{4}\defeq\{e,a,b,c\}\) die de volgende vermenigvuldiginstabel heeft:
\begin{center}
    \begin{tabular}{c|cccc}
              & \(e\) & \(a\) & \(b\) & \(c\)  \\\hline
        \(e\) & \(e\) & \(a\) & \(b\) & \(c\)  \\
        \(a\) & \(a\) & \(e\) & \(c\) & \(b\)  \\
        \(b\) & \(b\) & \(d\) & \(e\) & \(a\)  \\
        \(c\) & \(c\) & \(c\) & \(a\) & \(e\).
    \end{tabular}
\end{center}
Elk element van \(V_{4}\), behalve de eenheid, heeft orde \(2\), want \(x\cdot x=e\) voor alle \(x\in V_{4}\).

Deze groep kan ook geschreven worden als ondergroep van \(S_{4}\), namelijk als volgt:
\[
    V_{4}=\{(),(1\>2)(3\>4),(1\>3)(2\>4),(1\>4)(2\>3)\}.
\]

Voor de automorfismegroep van \(V_{4}\) geldt dat \(\Aut(V_{4})\cong S_{3}\), want de elementen \(a,b,c\) kunnen gepermuteerd worden.

\subsection{Quaternionengroep}
De ondergroep \(Q\defeq\{1,-1,i,-i,j,-j,k,-k\}\subseteq\h^{*}\) is een niet abelse ondergroep van \(\h^{*}\) van acht elementen.

\subsection{Symmetriegroep}
Zij \(X\) een verzameling. Dan is de groep \(S(X)\defeq\{f\colon X\to X\mid\text{\(f\) is een bijectie}\}\) een groep met als bewerking functiesamenstelling. De elementen van een \(S(X)\) heten permutaties. We noteren \(S_{n}\defeq S(\{1,\dots,n\})\). Deze verzameling heeft \(n!\) elementen.

\paragraph{De tekenfunctie}
Zij \(\sigma\in S_{n}\), dan is de volgende functie een homomorfisme:
\[
    P\colon S_{n}\to \text{GL}_{n}(n,\r)
\]

met \(P(\sigma)_{ij}=\delta_{i,\sigma(j)}\).

Aangezien de determinant \(\det\) ook een homomorfisme is met \(\text{GL}_{n}(n,\r)\to\r\), geldt dat \(\varepsilon\defeq\det\circ P\) ook een homomorfisme is met als beeld \(\{1,-1\}\). We noemen \(\varepsilon(\sigma)\) het teken van \(\sigma\in S_{n}\). De kern \(A_{n}\defeq\ker(\varepsilon)\) is een normaaldeler van grootte \(\frac{n!}{2}\).

\paragraph{Commutatorondergroepen} Zij \(n\in\n\), dan geldt voor \(S_{n}\) dat
\[
    [S_{n},S_{n}]=A_{n},
\]
en dat
\[
    [A_{n},A_{n}]=\begin{cases}
        \{(1)\} & n\in\{1,2,3\} \\
        V_{4}   & n=4           \\
        A_{n}   & n\geq5
    \end{cases}
\]

\subsection{Diëdergroep}
Beschouw een regelmatige \(n\)-hoek. Dan is \(D_{n}\) gedefinieerd als de congruenties die de \(n\)-hoek in zichzelf overvoeren met de bewerking van samenstelling. De groep \(D_{n}\) heeft orde \(2n\).

De groep \(D_{n}\) bestaat uit rotaties en spiegelingen:
\[
    D_{n}\defeq\left\{\rho_{0},\rho_{\frac{\pi}{n}},\dots,\rho_{\frac{2n-2}{n}\pi}\right\}\cup\left\{\sigma_{0},\dots,\sigma_{\frac{n-1}{n}\pi}\right\}.
\]

Voor het gemak wordt geschreven \(\rho\defeq\rho_{0}\) en \(\sigma\defeq\sigma_{0}\). Dan is \(\rho_{\frac{k}{n}\pi}=\rho^{k}\) en \(\sigma_{\frac{k}{n}\pi}=\rho^{k}\sigma\).

Dus \(D_{n}=\{\rho^{k}\mid k\in\n\}\cup\{\rho^{k}\sigma\mid k\in\n\}\). Ook gelden de volgende rekenregels:
\begin{align*}
    \rho^{n}=\rho,                   \\
    \sigma^{2}=\Id_{X},              \\
    \sigma\rho^{k}=\rho^{n-k}\sigma. \\
\end{align*}

\paragraph{Diëdergroep \(D_{n}\) met even \(n\)} Het centrum \(Z(D_{n})\) heeft twee elementen \(Z(D_{n})=\left\{e,\rho^{\frac{n}{2}}\right\}\) en de commutatorgroep is \(\gen{\rho^{2}}\). De conjugatieklassen zijn als volgt:
\begin{itemize}
    \item \(\{e\}\),
    \item \(\left\{\rho^{\frac{n}{2}}\right\}\),
    \item \(\frac{n}{2}-1\) klassen van de vorm \(\left\{\rho^{i},\rho^{-i}\right\}\),
    \item \(\left\{\rho^{1}\sigma,\rho^{3}\sigma,\dots,\rho^{n-1}\sigma\right\}\),
    \item \(\left\{\sigma,\rho^{2}\sigma,\dots,\rho^{n-2}\sigma\right\}\).
\end{itemize}

\paragraph{Diëdergroep \(D_{n}\) met oneven \(n\)} Deze groep heeft als centrum \(Z(D_{n})=\{e\}\), en de commutatorgroep is \([G,G]=\gen{\rho}\). Het heeft de volgende conjugatieklassen:
\begin{itemize}
    \item \(\{e\}\),
    \item \(\frac{n-1}{2}\) klassen van de vorm \(\left\{\rho^{i},\rho^{-i}\right\}\),
    \item \(\left\{\sigma,\rho^{1}\sigma,\dots,\rho^{n-1}\sigma\right\}\).
\end{itemize}

\subsection{Een tabel van alle groepen met orde \texorpdfstring{\(<16\)}{<16}}
De groepen die hier staan zijn alle groepen op isomorfisme met \(\#G<16\).
\begin{center}
    \begin{tabular}{c|l|l}
        Orde & Abels                                   & Niet abels        \\\hline
        1    & \(C_{1}\)                               & -                 \\
        2    & \(C_{2}\)                               & -                 \\
        3    & \(C_{3}\)                               & -                 \\
        4    & \(C_{4}, V_{4}\)                        & -                 \\
        5    & \(C_{5}\)                               & -                 \\
        6    & \(C_{6}\)                               & \(S_{3}\)         \\
        7    & \(C_{7}\)                               & -                 \\
        8    & \(C_{8},C_{4}\times C_{2},(C_{2})^{3}\) & \(Q,D_{4}\)       \\
        9    & \(C_{9},C_{3}\times C_{3}\)             &                   \\
        10   & \(C_{10}\)                              & \(D_{5}\)         \\
        11   & \(C_{11}\)                              & -                 \\
        12   & \(C_{12},C_{2}\times C_{6}\)            & \(D_{6},A_{4},B\) \\
        13   & \(C_{13}\)                              & -                 \\
        14   & \(C_{14},\)                             & \(D_{7}\)         \\
        15   & \(C_{15}\)                              &
    \end{tabular}
\end{center}