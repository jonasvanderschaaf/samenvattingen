\secnewpage{Isomorfie- en homomorfiestellingen}
\paragraph{De homomorfiestelling} Zij \(G_{1},G_{2}\) groepen en \(f\colon G_{1}\to G_{2}\) een homomorfisme en \(N\triangleleft G\) met \(N\subseteq\ker(f)\). Dan is er een unieke \(g\colon G_{1}/N\to G_{2}\) met \(g(\overline{a})\defeq f(a)\) zodat geldt dat \(g\circ\varphi=f\). Bovendien geldt dat \(\ker(g)=\ker(f)/N\subseteq G/N\).

\paragraph{Eerste isomorfiestelling} Zij \(G_{1},G_{2}\) groepen met \(f\colon G_{1},G_{2}\) een homomorfisme, dan geldt dat
\[
    G_{1}/\ker(f)\cong f(G_{1}).
\]

\subparagraph{Surjectief homomorfisme} Als de eisen hierboven gelden en \(f\) surjectief is, dan geldt dat \(G_{1}/\ker(f)\cong f(G_{1})=G_{2}\) met een isomorfie gegeven door \(a\cdot\ker(f)\mapsto f(a)\).

\paragraph{Homomorfisme tussen abelse groep} Zij \(G,A\) groepen en \(A\) abels. Dan is er voor elk homomorfisme \(f\colon G\to A\) een eenduidig bepaald homomorfisme \(g\colon G_{ab}\to A\) waarvoor geldt dat \(f=g\circ \varphi\). Hier is \(\varphi\colon G\to G_{ab}\) de canonieke afbeelding.

\paragraph{Doorsnedes van normaaldelers} Zij \(G\) een groep, \(N\triangleleft G\) en \(H\subseteq G\) een ondergroep. Dan geldt dat:
\begin{enumerate}
    \item \(N\cap H\triangleleft H\),
    \item \(HN=\{hn\mid h\in H,n\in N\}\) is een ondergroep van \(G\),
    \item \(H/(H\cap N)\cong HN/N\) (dit is de tweede isomorfiestelling).
\end{enumerate}

\paragraph{Derde isomorfiestelling} Zij \(G\) een groep en \(N,N'\triangleleft\) met \(N\subseteq N'\). Dan geldt dat \(N'/N\) een normaaldeler is van \(G/N\) en elke normaaldeler van \(G/N\) is van deze vorm. Ook geldt dat
\[
    (G/N)(N'/N)\cong G/N'.
\]