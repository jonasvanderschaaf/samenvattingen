\section{Toetsen}
\subsection{Hypotheses en toetsingsgrootheid}
\paragraph{Nul en alternatieve hypothese} Zij \(X\) een waarneming afhankelijk
van parameter \(\theta\in\Theta\), en \(H_{0},H_{1}\) een disjuncte partitie van
\(\Theta\). Dan heet \(H_{0}:\theta\in\Theta_{0}\) de nulhypothese, en
\(H_{1}:\theta\in\Theta_{1}\) de alternatieve hypothese.

\paragraph{Statistische toets} Gegeven een waarneming en nulhypothese \(H_{0}\),
en een verzameling \(K\subseteq\mathcal{X}\) met \(\mathcal{X}\) de
uitkomstenruimte van \(X\). Als \(H_{0}\) verworpen wordt precies als \(x\in K\),
dan noemen we \(K\) het kritieke gebied. Dan zeggen we dat \(K\) een statistische
toets bij \(H_{0}\) is.

\subparagraph{Toetsingsgrootheid} Vaak wordt een kritiek gebied gegeven door een
toetsingsgrootheid \(T=T(X)\) en een verzameling \(K_{T}\), dan wordt het
kritieke gebied gegeven door de deelverzameling van \(\mathcal{X}\):
\(K=\{X\in\mathcal{X}\mid T(X)\in K_{T}\}\).

\subsection{Onderscheidend vermogen van een toets}
\paragraph{Onderscheidend vermogen} Gegeven een nulhypothese
\(H_{0}:\theta\in\Theta_{0}\) en een kritiek gebied \(K\), is het
onderscheidend vermogen van de toets \(K\) gegeven door de functie
\[
    \theta\mapsto P_{\theta}(X\in K)=\pi(\theta;K).
\]

\paragraph{Onbetrouwbaarheid} De onbetrouwbaarheid van een toets \(K\) wordt
gegeven door
\[
    \alpha=\sup_{\theta\in\Theta_{0}}\pi(\theta;K).
\]
Een toets is van niveau \(\alpha_{0}\) als \(\alpha\leq\alpha_{0}\).

\subparagraph{Onbetrouwbaarheidsdrempel} Gegeven een vast getal \(\alpha_{0}\),
genaamd de onbetrouwbaarheidsdrempel, worden alleen toetsen \(K\) geaccepteerd
zodat \(\sup_{\theta\in\Theta_{0}}\pi(\theta;K)\leq\alpha_{0}\).

\subparagraph{Toets kiezen} Gegeven een onbetrouwbaarheidsdrempel \(\alpha_{0}\),
wordt de voorkeur gegeven aan een toets \(K\) zodat \(\pi(\theta;K)\) maximaal
is onder \(\theta\in\Theta_{1}\).

\subsection{Overschrijdingskansen}
De vorm van een kritiek gebied bepaalt ook de definitie van de
overschrijdingskans. Als \(K\) van de vorm \(\{x\mid T(X)\geq d\}\) is dan is de
overschrijdingskans gegeven door
\[
    \sup_{\theta\in\Theta_{0}}\p_{\theta}(T\geq t).
\]
Als \(K\) van de vorm \(\{x\mid T(X)\leq d\}\) is dan is de
overschrijdingskans gegeven door
\[
    \sup_{\theta\in\Theta_{0}}\p_{\theta}(T\leq t).
\]
Als \(K\) van de vorm \(\{x\mid T(X)\leq d\}\cup\{x\mid T(X)\geq d\}\) is dan is
de overschrijdingskans gegeven door
\[
    2\min\left(\sup_{\theta\in\Theta(0)}\p_{\theta}(T\leq t),
    \sup_{\theta\in\Theta(0)}\p_{\theta}(T\geq t)\right)
\]

\subsection{Standaard toetsen}
\subsubsection{Gauss toets}
Gegeven een steekproef \(X_{1},\dots,X_{n}\sim \mathcal{N}(\mu,\sigma^{2})\)
i.i.d., \(\mu\) onbekend en nulhypothese \(H_{0}:\mu\leq\mu_{0}\), geeft de
toetsingsgrootheid \(T(X)=\sqrt{n}\frac{\avg{X}-\mu_{0}}{\sigma}\) samen met
\(K_{T}=[\xi_{1-\alpha_{0}},\infty)\) het kritieke gebied dat een toets vormt
met onbetrouwebaarheid \(\alpha_{0}\).

\subsubsection{Chikwadraat en \texorpdfstring{\(t\)}{t}-verdeling}
\paragraph{Chikwadraat verdeling} Een stochast \(W\) heet \(\chi^{2}_{n}\)
verdeeld als \(W\sim\sum_{i=1}^{n}Z_{i}^{2}\) met \(Z_{i}\) een steekproef
uit \(\mathcal{N}(0,1)\). Dit wordt ook een Chikwadraat verdeling met \(n\)
vrijheidsgraden genoemd.

\paragraph{\texorpdfstring{\(t\)}{t}-verdeling} Zij \(Z\sim\mathcal{N}(0,1)\) en
\(W\sim\chi^{2}_{n}\) onafhankelijk, dan heeft \(\frac{Z}{\sqrt{\frac{W}{n}}}\)
een \(t_{n}\)-verdeling.

\paragraph{Normaalverdelingen en toetsen} Zij \(X_{1},\dots,X_{n}\) een
steekproef uit \(\mathcal{N}(\mu,\sigma^{2})\) dan geldt dat
\begin{enumerate}
    \item \(\avg{X}\sim\mathcal{N}(\mu,\frac{\sigma}{n})\),
    \item \(\frac{(n-1)S_{X}^{2}}{\sigma^{2}}\sim\chi^{2}_{n-1}\),
    \item \(\avg{X}\) en \(S_{X}^{2}\) zijn onafhankelijk,
    \item \(\sqrt{n}\frac{\avg{X}-\mu}{S_{X}}\sim t_{n-1}\).
\end{enumerate}

\subsubsection{Eensteekproeftoetsen}
\paragraph{\texorpdfstring{\(t\)}{t}-Toets} Zij \(X_{1},\dots,X_{n}\) een
steekproef uit \(\mathcal{N}(\mu,\sigma^{2})\), met \(\mu,\sigma^{2}\) onbekend.
Voor een nulhypothese \(H_{0}:\mu\leq\mu_{0}\), vormt de toetsingsgrootheid
\(T(X)=\sqrt{n}\frac{\avg{X}-\mu_{0}}{S_{X}}\) samen met het kritieke gebied
\([t_{n-1,1-\alpha_{0}},\infty)\) een toets van niveau \(\alpha_{0}\).

\paragraph{Toetsen voor \texorpdfstring{\(\sigma^{2}\)}{s^2}} Gegeven een
steekproef \(X_{1},\dots,X_{n}\) uit \(\mathcal{N}(\mu,\sigma^{2})\) en een
nulhypothese \(H_{0}:\sigma\leq\sigma_{0}\). Dan geldt dat de toetsingsgrootheid
\(T(X)=\frac{(n-1)S_{X}}{\sigma_{0}^{2}}\) en het kritieke gebied
\([\chi^{2}_{n-1,1-\alpha_{0}})\) eent toets van niveau \(\alpha_{0}\) vormt.

\subsubsection{Tweesteekproeftoetsen}
\paragraph{Gepaarde waarnemingen en \texorpdfstring{\(t\)}{t}-toetsen}
Gegeven gepaarde steekproeven \((X_{1},Y_{1}),\dots,(X_{n},Y_{n})\). Met
\(X_{1},\dots,X_{n}\) een steekproef uit \(\mathcal{N}(\mu,\sigma_{1}^{2})\) en
\(Y_{1},\dots,Y_{n}\) uit \(\mathcal{N}(\mu-\Delta,\sigma_{2}^{2})\).

Dan is \(Z_{i}=X_{i}-Y_{i}\) verdeeld met
\(\mathcal{N}(\Delta,\sigma_{1}^{2}+\sigma_{2}^{2})\). Hier kan dan de
\(t\)-toets op toegepast worden.

\paragraph{Twee steekproeven \texorpdfstring{\(t\)}{t}-toets} Gegeven twee
steekproeven \(X_{1},\dots,X_{m}\) en \(X_{1},\dots,X_{n}\) uit de verdelingen
\(\mathcal{N}(\mu,\sigma^{2})\) en \(\mathcal{N}(\nu,\sigma^{2})\), neem de
nulhypothese als \(H_{0}:\mu-\nu\leq0\), dan is de toetsingsgrootheid
\[
    T(X,Y)=\frac{\avg{X}-\avg{Y}}{S_{X,Y}\sqrt{\frac{1}{m}+\frac{1}{n}}},
\]
met
\[
    S_{X,Y}=\frac{1}{m+n-2}\left(\sum_{i=1}^{m}(X_{I}-\avg{X})^{2}
    +\sum_{j=1}^{n}(Y_{j}-\avg{Y})^{2}\right),
\]
samen met het kritieke gebied \([t_{m+n-2,1-\alpha_{0}},\infty)\) een toets van
niveau \(\alpha_{0}\). De schatter \(S_{X,Y}^{2}\sim\chi^{2}_{n+m-2}\) is een zuivere schatter voor
\(\sigma^{2}\)

\subsubsection{Aanpassingstoetsen}
Een aanpassingstoets is een type toets dat aangeeft of een kansverdeling lijkt
op een andere kansverdeling.

\paragraph{Kolmogorov-Smirnov-toets}
Gegeven een steekproef \(X_{1},\dots,X_{n}\) uit een onbekende verdeling \(F\).
Neem de nulhypothese \(H_{0}:F=F_{0}\). Dan is de Kolmogorov-Smirnov-toets
gebaseerd op de empirische verdelingsfunctie
\[
    \mathbb{F}_{n}(x)=\frac{1}{n}\#(X_{i}\leq x)=
    \frac{1}{n}\sum_{i=1}^{n}\mathbbm{1}_{X_{i}<x}
\]
Dan wordt de Kolmogorov-Smirnov-statistiek gegeven door
\[
    T=\sup_{x\in\r}\left\|\mathbb{F}_{n}(x)-F_{0}(x)\right\|.
\]
De nulhypothese wordt verworpen voor grote waarden van \(T\).

\subsection{Likelihood-ratiotoetsen}
Gegeven een kansdichtheid \(p_{\theta}\) van een stochastische vector \(X\) is
de likelihood ratiostatistiek gegeven een nulhypothese \(H_{0}:\theta\in\Theta_{0}\)
gedefinieerd als
\[
    \lambda(X)=\frac{\sup_{\theta\in\Theta}p_{\theta}(X)}
    {\sup_{\theta_{0}\in\Theta_{0}}p_{\theta_{0}}(X)}=
    \frac{p_{\hat{\theta}}(X)}{p_{\hat{\theta}_{0}}(X)},
\]
met \(\hat{\theta}\) de maximum likelihood schatter en \(\hat{\theta}_{0}\) de
maximum likelihood schatter beperkt tot \(\Theta_{0}\).

\subsection{Verdeling van \texorpdfstring{\(\lambda\)}{ll}} Onder
regulariteitsvoorwaarden geldt dat
\[
    2\log\lambda_{n}(X_{1},\dots,X_{n})\rightsquigarrow\chi^{2}_{k-k_{0}}
\]
met \(k\) de dimensie van \(\sqrt{n}(\Theta-\theta_{0})\) en \(k_{0}\) de
dimensie van \(\sqrt{n}(\Theta_{0}-\theta_{0})\) in een omgeving van
\(\theta_{0}\).

Als aan deze voorwaarden voldaan wordt dan geeft het verwerpen van \(H_{0}\) als
\(2\log\lambda\geq\chi^{2}_{k-k_{0},1-\alpha_{0}}\) een toets van niveau
\(\alpha_{0}\).