\subsection{Monotone rijen}

\paragraph{Wat is een monotone rij?} Een rij is monotoon als het één van de volgende twee is:

\subparagraph{Monotoon stijgend} Een rij $s_{n}\in\sequenceset$ is monotoon stijgend als $s_{n} \leq s_{n+1}$ voor elke $n\in\mathbb{N}$.

\subparagraph{Monotoon dalend} De rij $s_{n}\in\sequenceset$ is monotoon dalend als $s_{n+1} \leq s_{n}$ voor elke $n\in\mathbb{N}$.

\paragraph{Stellingen over monotoniciteit} De volgende stellingen over monotone rijen zijn waar:

\subparagraph{Gebonden monotone rijen} Elke begrensde monotone rij convergeert. \textit{Het bewijs hiervoor staat op pagina $57$ van het boek}.

\subparagraph{Onbegrense monotone rijen} Elke onbegrensde monotone rij divergeert naar $\pm\infty$. \textit{Het bewijs hiervoor staat op pagina $59$}.

\subparagraph{Limieten van monotone rijen} Voor elke monotone rij $s_{n}$ geldt dat $s_{n}$ convergeert of divergeert naar $\pm\infty$.

\subsection{Cauchy rijen}

\paragraph{Cauchy rijen} Een rij is Cauchy dan en slechts dan als voor elke $\epsilon>0$ er een getal $N$ bestaat zodat voor elke $m,n>N$ geldt dat $|s_{n}-s_{m}|<\epsilon$.

\paragraph{Convergentie en Cauchy} De volgende stellingen zijn waar over de convergentie en het Cauchy zijn van een rij.

\subparagraph{Convergente rij is Cauchy} Als een rij $s_{n}$ convergeert, dan is de rij ook Cauchy. \textit{Het bewijs hiervoor staat op pagina $63$}.

\subparagraph{Begrensdheid en Cauchy} Zij $s_{n}\in\sequenceset$. Als $s_{n}$ Cauchy is, dan is deze ook begrensd. \textit{Het bewijs hiervoor staat op pagina $63$}.

\subparagraph{Cauchy rij is convergent} Als een rij $s_{n}$ Cauchy is, dan convergeert deze. \textit{Het bewijs hiervoor staat in \hyperref[sec:AB]{appendix B}}.