\subsection{Afgeleides}

\paragraph{Wat is de afgeleide} Een functie $f:\mathbb{R}\to\mathbb{R}$ is differentieerbaar dan en slechts dan als de limiet $\lim{x}{a}\frac{f(x)-f(a)}{x-a}$ bestaat en eindig is. De waarde van de limiet is dan de afgeleide op dat punt. Op elk punt $a$ waar $f$ differentieerbaar is, is $f'(a)$ gedefinieerd als $\lim{x}{a}\frac{f(x)-f(a)}{x-a}$.

\subparagraph{Het domein van $f'$} Omdat $\text{dom}(f')$ de verzameling punten is waarop $f$ differentieerbaar is, geldt dat $\text{dom}(f')\subseteq\text{dom}(f)$.

\paragraph{Afgeleides en continuïteit} Als een functie differentieerbaar is op een punt $a$, dan is de functie ook continu. \textit{Het bewijs hiervoor staat op pagina $225$}.

\paragraph{Eigenschappen van de afgeleide} Zij $f$ en $g$ differentieerbare functies op een punt $a$, dan gelden de volgende stellingen:

\subparagraph{Het bepalen van de afgeleide} Om de afgeleide te bepalen zijn de volgende regels. \textit{Het bewijs hiervoor staat op pagina $226-227$}.


\begin{enumerate}
  \setlength\itemsep{0em}
  \item $(cf)'(a)=c\cdot f'(a)$
  \item $(f+g)'(a)=f'(a)+g'(a)$
  \item Voor het vermenigvuldigen van functies is de productregel: $(fg)'(a)=f'(a)g(a)+f(a)g'(a)$
  \item Voor het berekenen van het quotient van twee functies is de quotientregel: als $f(a)\neq0$, dan geldt dat $(\frac{f}{g})'(a)=\frac{g(a)f'(a)-g'(a)f(a)}{g^{2}(a)}$
\end{enumerate}

\paragraph{Kettingregel} Zij $f$ een op $a$ differentieerbare functie en $g$ op $f(a)$, dan geldt dat de samengestelde functie $g\circ f$ differentieerbaar is op $a$ en als afgeleide heeft het $(f\circ g)'(a) = g'(f(x))f'(a)$.
