\subsecnewpage{Oneigenlijke integralen}
\paragraph{\bq Eenzijdige\eq oneigenlijke integralen}

\subparagraph{Oneigenlijke integraal van links} Zij $f$ een integreerbare functie op $[a,b)$ ($b$ kan zowel eindig als oneindig zijn), dan geldt dat $\int_{a}^{b}f\defeq\uplim{c}{b}\int_{a}^{c}f$.

\subparagraph{Oneigenlijke integraal van rechts} Zij $f$ een integreerbare functie op $(a,b]$ ($a$ kan zowel eindig als min oneindig zijn), dan geldt dat $\int_{a}^{b}f\defeq\downlim{c}{a}\int_{c}^{b}f$.

\subparagraph{Oneigenlijke integraal aan beide kanten} Zij $I=(a,b)$ een interval zodat de functie $f:\mathbb{R}\to\mathbb{R}$ integreerbaar is op elk interval $[c,d]\subseteq I$. Dan geldt dat $\int_{a}^{b}f=\int_{a}^{\alpha}f+\int_{\alpha}^{b}$ voor een willekeurige $\alpha\in I$ als de integralen aan de rechterkant bestaan en de som niet van de vorm is $\infty-\infty$.

\paragraph{Cauchy-hoofdwaarde} Zij $f:\mathbb{R}\to\mathbb{R}$ een functie zodat de integraal $\int_{-\infty}^{\infty}f$ niet bestaat, maar $\liminfty{a}\int_{a}^{-a}f$ wel bestaat en een waarde $S$ heeft. Dan geldt dat de Cauchy-hoofdwaarde van de integraal $S$ is. \medskip

Een voorbeeld van zo'n functie is $f(x)=\text{sin}(x)$. We weten dat $\int_{0}^{\infty}\text{cos}(x)dx$ niet bestaat, maar $\int_{-a}^{a}\text{cos}(x)dx=0$, dus $\liminfty{a}\int_{-a}^{a}\text{cos}(x)dx=0$, dus de Cauchy-hoofdwaarde van $\text{cos}$ is $0$.

\paragraph{willekeurige machten} Voor een $b\in\mathbb{R}_{>0}$ geldt de volgende definitie $b^{x}\defeq E(xL(b))$. Omdat $L(e)=1$ geldt dat $e^{x}=E(x)$.
