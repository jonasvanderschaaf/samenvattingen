\subsection{Middenwaardestelling}

\paragraph{Maxima, minima en afgeleiden} Als $f$ gedefinieerd is op een open interval $(a,b)$ en een maximum aanneemt op $x_{0}$ en $f$ differentieerbaar is op $x_{0}$, dan geldt dat $f'(x_{0})=0$. \textit{Het bewijs hiervoor staat op pagina's $232$ en $233$}.

\paragraph{Stelling van Rolle} Zij $f$ een continue functie op $[a,b]$ die differentieerbaar is op $(a,b)$ met $f(a)=f(b)$. Dan is er tenminste één $x\in(a,b)$ zodat $f'(x)=0$.

\paragraph{Middenwaardestelling} Zij $f$ een continue functie op $[a,b]$ die differentieerbaar is op $(a,b)$. Dan is er tenminste één $x\in(a,b)$ zodat geldt dat $f'(x)=\frac{f(b)-f(a)}{b-a}$. \textit{Het bewijs hiervoor staat in \hyperref[sec:AE]{appendix E}}.
