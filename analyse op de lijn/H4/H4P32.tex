\subsection{Integratie}
\paragraph{Definitie van de integraal} De definitie van de integraal vereist bepaalde kennis, die stap voor stap hier wordt toegelicht:
\subparagraph{Maximum en minimum op een verzameling} Zij $f$ een functie gedefinieerd op de verzameling $[a,b]$, en zij $S\subseteq[a,b]$. Dan zijn de functies $M(f,S)$ en $m(f,S)$ als volgt gedefinieerd:

\begin{itemize}
  \setlength\itemsep{0em}
  \item $M(f,S)\defeq\text{sup}\{f(x):x\in S\}$
  \item $m(f,S)\defeq\text{inf}\{f(x):x\in S\}$
\end{itemize}

\subparagraph{Partities} Een partitie van een interval $[a,b]$ is een eindige verzameling van de vorm $P=\{a=t_{0}<t_{1}<...<t_{n}=b\}$.

\subparagraph{Darboux sommen} De boven Darboux som $U(f,P)$ is gedefinieerd als volgt: $U(f,P)\defeq\\\sum\limits_{k=1}^{n}M(f,[t_{k-1},t_{k}])\cdot(t_{k}-t_{k-1})$. De onder Darboux som is als volgt gedefinieerd: $L(f,P)\defeq\\\sum\limits_{k=1}^{n}m(f,[t_{k-1},t_{k}])\cdot(t_{k}-t_{k-1})$.

\subparagraph{Darboux integralen} De boven Darboux integraal van $f$ over $[a,b]$ $U(f)$ is gedefinieerd op de volgende wijze: $U(f)\defeq\text{inf}\{U(f,P):P\text{ is een partitie van }[a,b]\}$. Voor de onder Darboux integraal geldt dat $L(f)\defeq\text{sup}\{L(f,P):P\text{ is een partitie van }[a,b]\}$

\subparagraph{Integreerbaarheid} We zeggen dat een functie $f$ integreerbaar is op een interval $[a,b]$ als geldt dat $U(f)=L(f)$, dan geldt dat $\int_a^{b}f\defeq U(f)=L(f)$. Deze specifieke definitie van de integraal heet de \textbf{Darboux integraal}.

\paragraph{Stellingen over de Darboux integraal} De volgende stellingen over de Darboux integraal zijn waar:

\subparagraph{Deelpartities} Zij $f:\mathbb{R}\to\mathbb{R}$ een begrensde functie op $[a,b]$ en $P$ en $Q$ partities van $[a,b]$ zodat $P\subseteq Q$, dan geldt dat $L(f,P)\leq L(f,Q)\leq U(f,Q)\leq U(f,P)$. \textit{Het bewijs hiervoor staat op pagina $273$}.

\subparagraph{Boven- en ondergrenzen} Zij $f:\mathbb{R}\to\mathbb{R}$ een begrensde functie op $[a,b]$ en $P$ en $Q$ partities van $[a,b]$. Dan geldt dat $L(f,P)\leq U(f,Q)$. \textit{Het bewijs hiervoor staat op pagina $273$}.

\subparagraph{Boven en onderintegralen} Zij $f$ een begrensde functie op $[a,b]$, dan geldt dat $L(f)\leq U(f)$. \textit{Het bewijs hiervoor staat op pagina $274$}.

\subparagraph{Cauchy integratie} Een functie $f$ die begrensd is op $[a,b]$ is integreerbaar als en slechts als voor elke $\epsilon>0$ er een partitie $P$ van $[a,b]$ bestaat zodat $U(f,P)-L(f,P)<\epsilon$. \textit{Het bewijs hiervoor staat in \hyperref[sec:AF]{appendix F}}.

\paragraph{\textit{Mesh} van een partitie} De \textit{mesh} van een partitie is de maximale lengte van de deelintervallen van de partitie. Dus voor een partitie $P$ met $n$ deelintervallen geldt: $\text{mesh}(P)\defeq\text{max}\{t_{k}-t_{k-1}:k=1,2,...,n\}$.

\paragraph{Nog een Cauchy-criterium} Zij $f$ een begrensde functie op $[a,b]$, dan is $f$ integreerbaar dan en slechts dan als voor elke $\epsilon>0$ er een $\delta>0$ is zodat voor elke partitie $P$ van $[a,b]$ geldt dat als $\text{mesh}(P)<\delta$ dan geldt ook dat $U(f,P)-L(f,P)<\epsilon$. \textit{Het bewijs hiervoor staat op bladzijden $275$ en $276$}.

\paragraph{Riemann integratie} De Riemann integraal wordt gedefinieerd vanuit Riemann sommen, vervolgens wordt de integraal gedefinieerd.

\subparagraph{Riemann sommen} Zij $f$ een begrensde functie op $[a,b]$ en $P$ een partitie van $[a,b]$. Dan is een Riemann som een som van de vorm $\sum\limits_{k=1}^{n}f(x_{k})(t_{k}-t_{k-1})$. De rij $x_{k}$ is een willekeurig gekozen rij waarvoor geldt dat voor elke $k=1,2,...,n$ geldt dat $x_{k}\in[t_{k-1},t_{k}]$. Er zijn dus oneindig veel Riemann sommen voor een functie en partitie.

\subparagraph{Riemann integraal} Een functie $f:\mathbb{R}\to\mathbb{R}$ is Riemann integreerbaar dan en slechts dan als er een getal $r$ bestaat zodat voor elke $\epsilon>0$ er een $\delta>0$ is zodat voor elke Riemann som $S$ met bijbehorende partitie $P$ waarvoor geldt dat $\text{mesh}(P)<\delta$ geldt dat $|S-r|<\epsilon$. Dan zeggen we $\mathcal{R}\int_{a}^{b}f\defeq r$.

\paragraph{Riemann en Darboux integralen} Een functie $f:\mathbb{R}\to\mathbb{R}$ die begrensd is op $[a,b]$ is Riemann integreerbaar dan en slechts dan als $f$ Darboux integreerbaar is, dan komen de waarden van de twee integralen overeen. Dus{ $\int_{a}^{b}f=\mathcal{R}\int_{a}^{b}f$. \textit{Het bewijs hiervoor staat op pagina's $277$-$238$}.

\paragraph{Rij van Riemann sommen} Zij $f:\mathbb{R}\to\mathbb{R}$ een begrensde functie op $[a,b]$ en $(S)$ een rij Riemannsommen met voor elke $S_{n}$ bijbehorende partitie $P_{n}$ zodat $\liminfty{n}(\text{mesh}(P_{n}))=0$. Dan geldt dat $\liminfty{n}(S_{n})=\int_{a}^{b}f$. \textit{Het bewijs staat op pagina $278$-$279$}.
