\subsection{Uniforme continuïteit}

\paragraph{Wat is uniforme continuïteit} Zij $S$ een verzameling zodat $S\subseteq\mathbb{R}$. Een functie $f:S\to\mathbb{R}$ is uniform continu dan en slechts dan als voor elke $\epsilon>0$ er een $\delta>0$ zodat voor elke $x_{0},x \in S$ geldt dat als $|x_{0}-x|<\delta$ dan $|f(x_{0})-f(x)|<\epsilon$.

\subparagraph{Uniforme continuïteit vs continuïteit} Het verschil tussen de definitie van continuïteit en die van uniforme continuïteit is de plek waar de $x_{0}$ gedefiniëerd wordt. Bij uniforme continuïteit wordt de $x_{0}$ gedefiniëerd na de $\delta$, en bij \bq gewone\eq continuïteit wordt de $x_{0}$ gedefiniëerd voor de $\delta$. Dus $\delta$ kan ook afhangen van $x_{0}$ bij \bq gewone\eq continuïteit, wat niet kan bij uniforme continuïteit.

\paragraph{Stellingen over uniforme continuïteit} De volgende stellingen zijn waar over uniforme continuïteit:

\subparagraph{Continu op een interval en uniforme continuïteit} Als een functie $f$ continu is op een gesloten interval $[a,b]$, dan is $f$ uniform continu op $[a,b]$. \textit{Het bewijs hiervoor staat op pagina $143$}.

\subparagraph{Uniform continue functies en Cauchy rijen} Zij $f$ een uniform continue functie op een verzameling $S\in\text{dom}(f)$ en $s_{n}\in\sequenceset$ zodat $s_{n}$ ee Cauchy-rij is. Dan geldt dat $(f(s_{n}))$ ook een Cauchy-rij is. \textit{Het bewijs staat op pagina $146$}.

\subparagraph{Uitbreiden van domein} Zij $f:(a,b)\to\mathbb{R}$ een functie. Dan kan $f$ uitgebreid worden naar een continue functie $\widetilde{f}:[a,b]\to\mathbb{R}$ dan en slechts dan als $f$ uniform continu is op $(a,b)$. \textit{Het bewijs staat op bladzijde $148$-$149$}.