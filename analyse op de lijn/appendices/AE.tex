\subsecnewpage{Appendix E: de stelling van Rolle en de middelwaardestelling}
\label{sec:AE}
\paragraph{De stelling van Rolle} Zij $f$ een functie die continu is op $[a,b]$ en differentieerbaar op $(a,b)$ waarvoor geldt dat $f(a)=f(b)$, dan is er minstens één $x\in(a,b)$ waarvoor geldt dat $f'(x)=0$. Dit heet de stelling van Rolle.

\begin{proof}
  Omdat $f$ continu is weten we dat er $x_{0},y_{0}\in[a,b]$ zodat voor elke $x\in[a,b]$ geldt dat\\ $f(x_{0})\leq f(x)\leq f(y_{0})$. Dan zijn er twee gevallen:

  \begin{enumerate}
    \setlength\itemsep{0em}
    \item De punten $x_{0}$ en $y_{0}$ zijn beiden eindpunten, dan is de functie $f$ constant, en dan is de afgeleide overal $0$
    \item Anders neemt $f$ een maximum of minimum aan op $x_{0}$ of $y_{0}$, dus dan is op $x_{0}$ of op $y_{0}$ de afgeleide $0$.
  \end{enumerate}

  \noindent In beide gevallen geldt dan dat er een punt is waar de afgeleide $0$ is, dus we hebben de stelling van Rolle bewezen.
\end{proof}

\paragraph{middelwaardestelling} Zij $f$ een functie die continu is op $[a,b]$ die differentieerbaar is op $(a,b)$. Dan is er tenminste één $x\in(a,b)$ waarvoor geldt dat $f'(x)=\frac{f(b)-f(a)}{b-a}$. Dit heet de middelwaardestelling.

\begin{proof}

Zij $L$ een functie die een rechte lijn trekt tussen $(a,f(a))$ en $(b,f(b))$. Dan geldt dat $L(a)=f(a)$ en $L(b)=f(b)$. Ook geldt dat $L'(x)=\frac{f(b)-f(a)}{b-a}$ voor alle $x$. \medskip

\noindent Zij $g(x)=f(x)-L(x)$ voor $x\in[a,b]$, dan geldt dat $g(a)=0=g(b)$, dus volgens de stelling van Rolle geldt dan dat er een $x\in(a,b)$ is zodat $g'(x)=0$. En volgens de regels van afgeleides nemen geldt dat $g'(x)=L'(x)-f'(x)=\frac{f(b)-f(a)}{b-a}-f'(x)$. Hieruit volgt dan duidelijk dat $\frac{f(b)-f(a)}{b-a}=f'(x)$.

\end{proof}
