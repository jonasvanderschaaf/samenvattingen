\subsecnewpage{Appendix H: Hoofdstelling van de calculus 1}
\label{sec:AH}
\paragraph{Te bewijzen} Zij $f$ een continue functie op $[a,b]$ die differentiëerbaar is op $(a,b)$ en zij $f'$ integreerbaar op $[a,b]$, dan geldt dat $\int_{a}^{b}f'=f(b)-f(a)$.

\begin{proof}
  Zij $\epsilon>0$. Omdat $f'$ integreerbaar is, is er een partitie $P$ zodat $U(f',P)-L(f',P)<\epsilon$. \medskip

  \noindent Volgens de middenwaardestelling geldt dat in elk stuk van de partitie $[t_{k-1},t_{k}]$ dat er een $x_{k}$ is zodat $(t_{k}-t_{k-1})f'(x_{k})=f(t_{k})-f(t_{k-1})$. \medskip

  \noindent Daardoor geldt dat $f(b)-f(a)=\sum\limits_{k=1}^{n}(f(t_{k})-f(t_{k-1}))=\sum\limits_{k=1}^{n}f'(x_{k})(t_{k}-t_{k-1})$. Omdat geldt dat $m(f',[t_{k-1},t_{k}])\leq f'(x_{k})\leq M(f,[t_{k-1},t_{k}])$, geldt dat $L(f',P)\leq f(b)-f(a)\leq U(f',P)$. \medskip

  \noindent Daaruit volgt dan dat $|\int_{a}^{b}f'-(f(b)-f(a))|<\epsilon$. Omdat $\epsilon$ arbitrair is geldt dat $\int_{a}^{b}f'=f(b)-f(a)$.
\end{proof}
