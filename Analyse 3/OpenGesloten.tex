\section{Open, gesloten en compacte verzamelingen}

\subsection{Open verzamelingen}
\paragraph{Bollen} Zij \((S,d)\)\footnote{In de rest van deze samenvatting wordt in elke definitie geïmpliceerd dat \((S,d)\) een metrische ruimte is zonder dat het er explicitiet bij wordt gezegd.} een metrische ruimte. Dan geeft de functie 
\[
    B\colon S\times\r_{>0}\to\mathcal{P}(S)\colon (s_{0},r)\mapsto \{s\in S\colon d(s_{0},s)<r\}
\]
een bol met straal \(r\) en middelpunt \(s_{0}\).

\paragraph{Inwendige punten} Zij \(E\subseteq S\), dan geldt dat een punt \(s_{0}\in E\) inwendig is als geldt dat er een \(r>0\) is zodat
\[
    B(s_{0},r)\subseteq E.
\]
Voor de verzameling \(E\) geldt dan dat \(E^{\circ}\defeq\{s\in E\colon\text{\(s\) is inwendig}\}\). De verzameling \(E^{\circ}\) heet dan het inwendige van \(E\).

\paragraph{Open verzamelingen} Een verzameling \(E\subseteq S\) heet open als geldt dat \(E=E^{\circ}\).

\subparagraph{Stellingen over open verzamelingen} De volgende stellingen over open verzamelingen zijn\\ waar:
\begin{items}
    \item De verzameling \(S\) is open in \(S\)
    \item De lege verzameling \(\emptyset\) is open in \(S\)
    \item De vereniging van open verzamelingen is open\footnote{Dit geldt ook als je oneindig veel verzamelingen neemt.}.
    \item De doorsnede van \emph{eindig} veel open verzameling is een open verzameling.
\end{items}

\subsection{Gesloten verzamelingen}
\paragraph{Wat is een gesloten verzameling} We noemen een verzameling \(E\subseteq S\) gesloten als het complement \(S\setminus E\) open is. Dat is equivalent met het idee dat er een open verzameling \(U\in S\) is zodat \(E=S\setminus U\).

Merk op dat open en gesloten \emph{\textbf{niet}} tegenovergesteld aan elkaar zijn. Zo zijn bijvoorbeeld \(S\) en \(\emptyset\) open én gesloten, en bijvoorbeeld \([1,2)\) noch gesloten, noch open in \(\r\).

\paragraph{Afsluiting van een verzameling} Gegeven een verzameling \(E\in S\) is de afsluiting \(E^{-}=\overline{E}\) de doorsnede van alle gesloten verzamelingen die \(E\) bevatten. Als formule opgeschreven betekent dat dat gegeven een \(E\subseteq S\), en \(\mathcal{F}=\{F\subseteq S\colon\text{\(E\subseteq F\) en \(F\) is gesloten}\}\) geldt dat
\[
    \overline{E}\defeq\bigcap\limits_{F\in\mathcal{F}}F
\]
\subparagraph{Stellingen over gesloten verzamelingen} De volgende stellingen over gesloten verzamelingen zijn waar:
\begin{items}
    \item Een verzameling \(E\) is gesloten dan en slechts dan als \(E=\overline{E}\).
    \item Een verzameling \(E\) is gesloten dan en slechts dan als \(E\) alle limieten bevat van alle rijen in \(E\).
    \item Een element \(s\in S\) is in \(\overline{E}\) dan en slechts dan als het de limiet is van een rijtje in \(E\).
\end{items}

\paragraph{Rand van een verzameling} Voor een verzamelingen \(E\subseteq S\) is de rand \(\partial E\) als volgt\\ gedefiniëerd:
\[
    \partial E\defeq \overline{E}\setminus E^{\circ}
\]

\paragraph{Rijen van gesloten verzamelingen} Zij \(\downsequence{F}\) een rij van gesloten, begrensde niet-lege verzamelingen in \(\rk{k}\) die dalend is (voor alle \(i\in\n\) geldt dat \(F_{i}\supseteq F_{i+1}\)). Dan geldt dat
\[
    F=\bigcup\limits_{n=1}^{\infty}F_{n}
\]
ook gesloten, begrensd en niet leeg is.

\subsection{Compacte verzamelingen}
\paragraph{Open overdekkingen} Zij \(E\subseteq S\). Dan is de familie\footnote{Een familie is een collectie van verzamelingen.} \(\mathcal{U}\) van open verzamelingen in \(S\) een open overdekking als geldt dat
\[
    E\subseteq\bigcup\limits_{U\in\mathcal{U}}U.
\]
Een deeloverdekking van \(\mathcal{U}\) is een deelfamilie van \(\mathcal{U}\) die ook een overdekking is van \(E\).

Een overdekking is eindig als geldt dat deze slechts eindig veel elementen heeft.

\paragraph{Compactheid} Een verzameling \(E\subseteq S\) is compact als geldt dat elke open overdekking \(\mathcal{U}\) een eindige deeloverdekking heeft van \(E\).

\paragraph{Heine-Borel stelling} Een verzameling \(E\subseteq \rk{k}\) is compact dan en slechts dan als deze begrensd en gesloten is.

\paragraph{Compactheid van \(k\)-cellen} Elke \(k\)-cel \([a_{1},b_{1}]\times[a_{2},b_{2}]\times\sdots\times[a_{k},b_{k}]\) in \(\rk{k}\) is compact.